\documentclass[10pt,landscape,letterpaper]{article} % Copyright (c) 2020 Brian Schubert
\usepackage{multicol}
\usepackage[landscape]{geometry}
\usepackage{amsmath,amsfonts,amssymb}
\usepackage{mathtools}
\usepackage{fancyhdr}
\usepackage{enumitem}
\usepackage{tabularx}
\usepackage{titlesec}
\usepackage{array}

% Units for second page
\usepackage{siunitx}
\DeclareSIUnit{\atm}{atm}

%
% Styling inspired by tex.stackexchange user "Dror" in this posting:
% https://tex.stackexchange.com/questions/8827/preparing-cheat-sheets
%

\thispagestyle{fancy}
\cfoot{\footnotesize Copyright \textcopyright \ 2020 Brian Schubert}
\rfoot{\footnotesize Revised 2020-04-24}
\lhead{PHYS 1151\\Physics for Engineering 1}
\rhead{Spring 2020 \\ Prof. Matthew Webber}
%\chead{\large \underline{Midterm Exam ``Cheat Sheet"}}
\renewcommand{\headrulewidth}{0pt}
\setlength{\voffset}{-4pt}

\geometry{top=0.7in,left=.4in,right=.4in,bottom=.6in}


% Redefine section commands to use less space
% NOTE - still to be tweak to maximize space with reasonable readbability
\makeatletter
\renewcommand{\section}{\@startsection{section}{1}{0mm}%
                                {-1ex plus -.5ex minus -.2ex}%
                                {0.5ex plus .2ex}%x
                                {\normalfont\large\bfseries}}
\renewcommand{\subsection}{\@startsection{subsection}{2}{0mm}%
                                {-1explus -.5ex minus -.2ex}%
                                {0.5ex plus .2ex}%
                                {\normalfont\normalsize\bfseries}}
% Custom squishing for \paragraph
\renewcommand\paragraph{\@startsection{paragraph}{4}{\z@}%
    {0.5ex \@plus1ex \@minus.2ex}%
    {-1em}%
    {\normalfont\small\bfseries}}

% Don't print section numbers
\setcounter{secnumdepth}{0}



\setlength{\parindent}{0pt}
\setlength{\parskip}{0pt plus 0.5ex}

% Notation Definitions
\newcommand{\dd}[1]{\mathrm{d}#1}
\newcommand{\diffop}[2][]{\frac{\dd#1}{\dd #2}}
\newcommand{\pdiffop}[2][]{\frac{\partial #1}{\partial #2}}
\newcommand{\matr}[1]{\mathbf{#1}}
\newcommand{\bvec}[1]{{\vec{\mathbf{#1}}}}
\newcommand{\vecspace}[1]{\mathbb{#1}}
\newcommand{\transp}{^\mathrm{T}}

\newcommand\cheatsheetmargin{0.2cm}

% -----------------------------------------------------------------------

\begin{document}
\raggedright
\footnotesize
\begin{multicols*}{3}


% multicol parameters
% These lengths are set only within the two main columns
%\setlength{\columnseprule}{0.25pt}
\setlength{\premulticols}{1pt}
\setlength{\postmulticols}{1pt}
\setlength{\multicolsep}{1pt}
\setlength{\columnsep}{2pt}


\section{Vector Products}

\begin{gather*}
\bvec v \cdot \bvec w = \|\bvec v\|\|\bvec w\| \cos\theta = \|\bvec v\| \|\operatorname{proj}_\bvec{v} \bvec w\|  = \|\bvec w\| \|\operatorname{proj}_\bvec{w} \bvec v \| \\
\bvec v \times \bvec w = -( \bvec w \times \bvec v), \quad \| \bvec v \times \bvec w\| = \|\bvec v\|\|\bvec w\| \sin\theta\\
\end{gather*}

\vspace{-10pt}

\section{Kinematics}
\begin{multicols}{2}
    
    \begin{center}\textbf{Velocity and Linear Acceleration}\end{center}
    \begin{gather*}
    \bvec{v}_\mathrm{avg} = \frac{\Delta \bvec{x}}{\Delta t}, \qquad \bvec{v} = \diffop[\bvec x]{t}\\
    \bvec{x}_f - \bvec{x}_i = \Delta \bvec{x} = \int_{t_i}^{t_f} \bvec{v}(t)\,\mathrm dt
    \\
    \bvec{a}_\mathrm{avg} = \frac{\Delta \bvec{v}}{\Delta t},  \qquad \bvec{a} = \diffop[\bvec v]{t} = \diffop[^2\bvec x]{t^2}
    \end{gather*}
    
    \columnbreak
    
    \begin{center}\textbf{Under constant acceleration}\end{center}
    \begin{align*}
    v_f &= v_i + at\\
    \Delta x &= v_i t + \frac{1}{2} at^2\\
    v_f^2 &= v_i^2 + 2ax\\
    \end{align*}
\end{multicols}
\vspace{-5pt}
\subsection{Uniform Circular Motion}
\begin{gather*}
    a_\mathrm{rad} = \frac{v^2}{R} = \frac{4\pi^2 R}{T^2},\qquad v_\mathrm{rad} = \frac{2\pi R}{T}
\end{gather*}
\vspace{-18pt}
\section{Forces}
\begin{align*}
\sum \bvec{F} &= \bvec{0} \tag{Newton's first law at equilibrium}\\
\sum \bvec{F} &= m\bvec{a} \tag{Newton's second law}\\
\bvec{F}_{A/B} &= -\bvec{F}_{B/A} \tag{Newton's third law}
\end{align*}
\begin{equation*}
F_\mu = \mu_{k} F_N, \qquad F_\mu \leq \mu_s N \tag{Kinetic vs Static friction}
\end{equation*}
\section{Work}
\begin{align*}
W &= \bvec{F} \cdot \bvec {s} \tag{Constant force in a line}\\
W &= \Delta K \tag{Work-energy theorem} \\
W &= \int_C \bvec{F}(\bvec{x})\cdot\mathrm{d}\bvec{x} \tag{Varying force on currved path}
\end{align*}
\vspace{-10pt}
\paragraph{Power}
\begin{align*}
P_\mathrm{avg} = \frac{\Delta W}{\Delta t},& \quad P = \diffop[W]{t} \tag{Avg. and Instanteous Power}\\
P &= \bvec{F} \cdot \bvec{v} \tag{Instantaneous Power}
\end{align*}
\vspace{-20pt}
\section{Energy}
\begin{align*}
U_\mathrm{grav} &= mgh \tag{Gravational potential energy}\\
K &= \frac{1}{2}mv^2 \tag{Linear kinetic energy}\\
U_\mathrm{el} &= \frac{1}{2}kx^2 \tag{Eleastic potential energy}\\ 
\Delta U &= -\int_{x_i}^{x_f} \bvec{F}\cdot \mathrm{d}\bvec{x} \tag{Potential Energy from foce field}\\
F_x &= -\diffop[U]{x} \tag{Force from potential enegery}\\ 
\bvec{F} &= - \nabla U = -\left\langle \pdiffop[U]{x},\pdiffop[U]{y}, \pdiffop[U]{z} \right\rangle \tag{Force from potential energy}\\
\end{align*}

\columnbreak

\section{Momentum}
\begin{align*}
\bvec{p} &= m\bvec{v} \tag{Momenutm} \\
\sum \bvec{F} &= \diffop[\bvec{p}]{t} = m\diffop[\bvec{v}]{t} \tag{Newton's seond law}\\
\bvec{P} &= \sum m_i\bvec{v}_i = M\bvec{v}_\mathrm{cm} \tag{Total momentum of system}\\
\end{align*}

\vspace{-15pt}

\subsection{Impulse}
.\begin{align*}
\bvec{J} &= \bvec{p}_2 - \bvec{p}_1  = \Delta \bvec{p} \tag{Impulse} \\
\bvec{J} &= \sum\bvec{F}\Delta t = \bvec{F}_\mathrm{avg} \Delta t \tag{Impulse from constant external force} \\
\bvec{J} &= \int_{t_i}^{t_f} \bvec{F}(t) \,\mathrm{d}t \tag{Impulse from general external force}\\
\end{align*}

\vspace{-10pt}

\subsection{Collisions}
\begin{multicols}{2}
    \begin{center}
        \underline{Elastic Collisions}
    \end{center}
    \begin{itemize}[leftmargin=14pt]
        \item Kinetic energy and momentum conserved
        \item Bodies do not permanently deform
    \end{itemize}

    \columnbreak
    
   \begin{center}
        \underline{Inelastic Collisions}
    \end{center}
    \begin{itemize}[leftmargin=14pt]
        \item Momentum conserved
        \item Energy dissipated through sound, heat, deformation
        \item Perfectly inelastic - bodies remain in contact after collision ($\sum \bvec{p}_i = \bvec{0}$)
    \end{itemize}
\end{multicols}

\paragraph{Center of Mass}
\begin{align*}
\bvec{r}_\mathrm{cm} =\frac{\sum_i m_i \bvec{r}_i}{\sum_i {m_i}} \tag{Center of mass}\\
\sum{\bvec{F}_\mathrm{ext}} = M \bvec{a}_\mathrm{cm} \tag{Net external force}\\
\end{align*}


\section{Angular Kinematics}
\begin{multicols}{2}
    \begin{center}
    \textbf{Angular Velocity and Acceleration}
    \end{center}
    \begin{gather*}
    \omega_\mathrm{avg} = \frac{\Delta \theta}{\Delta t}, \qquad \omega = \diffop[ \theta]{t}\\
    \theta_f - \theta_i = \Delta \theta = \int_{t_i}^{t_f} \omega(t)\,\mathrm dt
    \\
    \alpha_\mathrm{avg} = \frac{\Delta \omega}{\Delta t},  \qquad 
    \alpha = \diffop[\omega]{t} = \diffop[^2 \theta]{t^2}
    \end{gather*}
    \columnbreak
    \begin{center}
        \textbf{Under constant acceleration}
    \end{center}
    \begin{align*}
    \omega_f &= \omega_i + \alpha t\\
    \Delta \theta &= \omega_i t + \frac{1}{2} \alpha t^2\\
    \omega_f^2 &= \omega_i^2 + 2\alpha x
    \end{align*}
    \vfill\null
\end{multicols}
\begin{align*}
v_\mathrm{tan} &= \diffop[s]{t} = r \diffop[\theta]{t} = r\omega \tag{Tangential and angular velocity}\\
a_\mathrm{tan} &= r\alpha \tag{Tangential and angular acceleration} \\
a_\mathrm{rad} &= \frac{v^2_\mathrm{tan}}{r} = \omega^2 r \tag{Radial and angular acceleration} \\
K_\mathrm{rot} &= \frac{1}{2} I\omega^2 \tag{Rotational kinetic energy}
\end{align*}

\columnbreak

\begin{minipage}{\columnwidth}
\section{Moment of Inertia}

\begin{align*}
I_\mathrm{pm} &= mr^2 \tag{Moment of intertia of point mass}\\
I &= \int r^2 \,\mathrm{d}m \tag{Moment of intertia of continous mass}\\
I &= \sum m_i r_i^2 \tag{Moment of intertia of collection of particles}\\
I &= I_\mathrm{cm} + Md^2 \tag{Parallel axis theorem}
\end{align*}
\vspace{-10pt}

\section{Torque and Angular Momentum}
\begin{align*}
\bvec{\tau} &= \bvec{r} \times \bvec{F}  \tag{Torque vector}\\
\tau &= r_\perp F = rF_\perp \tag{Magnitude of torque}\\
\sum \bvec{\tau} &= I \bvec{\alpha} \tag{Newton's second law}\\
\bvec{L} &= \bvec{r} \times \bvec{p} = \bvec{r}\times m\bvec{v} \tag{Angular momentum of particle}\\
\bvec{L} &= I \bvec{\omega} \tag{Angular momentum of rigid axis} \\
\bvec{\tau} &= \diffop[\bvec{L}]{t} \tag{Torque and angular momentum}\\
\end{align*}

\vspace{-12pt}

\paragraph{Energy and Rotation}
\begin{align*}
K &= \frac{1}{2}Mv^2_\mathrm{cm} + \frac{1}{2}I_\mathrm{cm}\omega^2 \tag{Kinetic energy of rigid body} \\
W &= \int_{\theta_i}^{\theta_f} \tau_z \,\mathrm{d}\theta \tag{Work done by $\tau_z$}\\
W &= \tau_z \Delta \theta \tag{Work done by constant $\tau_z$}\\
P &= \tau_z \omega_z \tag{Power due to torque}
\end{align*}

\section{Equilibrium}
\begin{equation*}
\sum\bvec{F} = \bvec{0}, \qquad \sum\bvec{\tau} = \bvec{0}
\end{equation*}

\begin{align*}
Y &= \frac{(\text{stress})}{(\text{strain)}} = \frac{F_\perp / A}{\Delta L / L_0} \tag{Young's modulus} \\
B &= \frac{(\text{stress})}{(\text{strain)}} = -\frac{\Delta p}{\Delta V/V0} =  -\frac{F_\perp / A}{\Delta L / L_0} \tag{Bulk modulus} \\
S &= \frac{(\text{stress})}{(\text{strain)}} = \frac{F_{|| / A}}{x/h} \tag{Shear modulus}
\end{align*}

\section{Fluid Statics}
\begin{align*}
\rho &= m/V \tag{Density}\\
P &= F/A \tag{Pressure}\\ 
F_B &= m_\mathrm{fluid} g = \rho_\mathrm{fluid}gV \tag{Archimedes Principle}\\
P_\mathrm{gauge} &= \rho g h \tag{Gauge pressure}\\
P_\mathrm{abs} &= P_0 + P_\mathrm{gauge} \tag {Absolute Pressure}\\
sg &= \rho / \rho_\mathrm{water} \tag{Specific gravity}
\end{align*}
\end{minipage}

\newpage

\thispagestyle{fancy}
\cfoot{\footnotesize Copyright \textcopyright \ 2020 Brian Schubert}
\rfoot{\footnotesize Revised 2020-04-24}
\lhead{PHYS 1151\\Physics for Engineering 1}
\rhead{Spring 2020 \\ Prof. Matthew Webber}

\section{Fluid dynamics}

\begin{align*}
\diffop[m]{t} &= \rho A v = \text{const.} \tag{Mass flow rate for ideal fluid}\\
\diffop[V]{t} &= Av = \text{const.} \tag{Volume flow rate for ideal fluid}\\
A_1 v_1 &= A_2 v_2 \tag{Continuity equation}\\
P_1 + \rho g y_1 &+ \frac{1}{2}\rho v_1^2 = P_2 + \rho g y_2 + \frac{1}{2}\rho v_2^2 = \text{const.} \tag{Bernouilli's Equation $(W + PE + KE)/V$}\\
\end{align*}

\section{Periodic Motion}

\begin{align*}
f &= \frac{1}{T} \tag{Frequency and period} \\
\omega &= 2\pi f = \frac{2\pi}{T} \tag{Angular frequency}\\
F_x &= -kx \tag{Restoring force for ideal spring}\\
a_x &= \diffop[^2 x]{t} = -\frac{k}{m}x \tag{Simple harmonic motion}\\
\omega &= \sqrt{\frac{k}{m}} \tag{Angular frequency for simple harmonic motion}\\
x &= A\cos(\omega t + \phi)\ \tag{Displacement in simple harmonic motion}\\
E &= \frac{1}{2}mv_x^2 +\frac{1}{2}kx^2 = \frac{1}{2} kA^2 = \text{const.} \tag{Total mechaical energy in simple harmonic motion}\\
\omega &= \sqrt{\frac{k}{m}} = \sqrt{\frac{mg/L}{m}} = \sqrt{\frac{g}{L}} \tag{Anuglar frequency of simple pendulum, small amplitude} \\
\omega &= \sqrt{\frac{mgd}{I}} \tag{Angular freqency of physical pendulum, small amplitude}\\
\end{align*}

\vfill\null


\columnbreak

\sisetup{inter-unit-product =\cdot}
\sisetup{per-mode = symbol}

\section{Units}
%\fbox{\parbox{\columnwidth}{
        \begin{align*}c
        \si{\newton} &= \si{\kg\meter\per\second\squared} \tag{Newtons}\\
        \si{\joule} &= \si{\kg\meter\squared\per\second\squared} \tag{Joules}\\
        \si{\watt} &= \si{\kg\meter\squared\per\second\cubed} \tag{Watts}\\
        \si{\pascal} &= \si{\kg\per\meter\per\second\squared} \tag{Pascals}
        \end{align*}
%}}
    
\section{Common Constants}
\begin{align*}
g &= \SI{9.81}{\meter\per\second\squared} \\
\SI{1}{\atm} &= \SI{101325}{\pascal}\\
\SI{1}{\atm} &= \SI{760}{\mmHg}
\end{align*}
\end{multicols*}
\end{document}